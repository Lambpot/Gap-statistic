\chapter{Conclusion}

\startcontents[Conslusion]
\section*{Conclusion et Limites de la méthode du Gap Statistique}

Dans ce travail, nous avons d'abord présenté la méthode du Gap Statistique développée par Tibshirani et al. (2000), avant de l'appliquer sur deux types de données : d'une part, des données simulées dont le nombre optimal de clusters est connu, et d'autre part, une base de données réelle. Nos résultats montrent que la méthode du Gap Statistique constitue un outil puissant pour déterminer le nombre de clusters dans un jeu de données lorsque la distribution sous-jacente est inconnue.

Contrairement à d'autres méthodes couramment proposées dans la littérature, qui peuvent se révéler inefficaces en présence d'une seule composante dans les données, le Gap Statistique offre une approche robuste permettant de vérifier la nécessité ou non d'un clustering.

Cependant, cette méthode présente certaines limites. En particulier, elle peut produire des résultats erronés lorsque la distribution des données n'est pas log-normale. Dans de tels cas, des alternatives comme celle proposée par Kaufman et Rousseeuw (1990) peuvent être plus efficaces, car elles ne reposent sur aucune hypothèse spécifique concernant la distribution des données.